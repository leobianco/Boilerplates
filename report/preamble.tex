% Preamble

% Graphics
\usepackage{xcolor}
\usepackage{graphicx}
\usepackage{tikz}
\usepackage{tikz-network}
\usetikzlibrary{
	mindmap, trees, arrows, arrows.meta, positioning, decorations.markings, 
	fit, decorations.pathmorphing
}
\usepackage{caption}
\usepackage{subcaption}  % grids of images
\usepackage[section]{placeins}  % figures stay in section
\usepackage[framemethod=tikz]{mdframed} % colored boxes

% Define custom colors
\definecolor{lightgreen}{RGB}{250,255,250}
\definecolor{emeraldgreen}{RGB}{80, 200, 120}

% Define colored boxes
\newmdenv[innerlinewidth=0.5pt, roundcorner=4pt,linecolor=red!40,backgroundcolor=red!10,innerleftmargin=6pt,
innerrightmargin=6pt,innertopmargin=0pt,innerbottommargin=\topsep]{redbox}
\newmdenv[innerlinewidth=0.5pt, roundcorner=4pt,linecolor=emeraldgreen,backgroundcolor=lightgreen,innerleftmargin=6pt,
innerrightmargin=6pt,innertopmargin=0pt,innerbottommargin=\topsep]{greenbox}
\newmdenv[innerlinewidth=0.5pt, roundcorner=4pt,linecolor=yellow,backgroundcolor=yellow!25,innerleftmargin=6pt,
innerrightmargin=6pt,innertopmargin=6pt,innerbottommargin=6pt]{yellowbox}
% The inner margins are set this way to be harmonious with the amsthm environments I created

% Mathematics and code
\usepackage{amsthm,amsmath}
\usepackage{flexisym}  % recommended for breqn
\usepackage{breqn}  % breaking long equations, load it before amssymb
\usepackage{amssymb, mathrsfs}
\usepackage{mathtools}  % declaring new paired delimiters
\usepackage{algorithm}
\usepackage{algpseudocode}
\usepackage{thmtools}  % theorem, definition, and question lists

\theoremstyle{plain}
\newtheorem{theorem}{Theorem}[section]
\newtheorem{proposition}[theorem]{Proposition}
\newtheorem{lemma}[theorem]{Lemma}
\newtheorem*{corollary}{Corollary}
\theoremstyle{definition}
\newtheorem{definition}{Definition}[section]
\newtheorem{example}{Example}[section]
\newtheorem{question}{Question}
\theoremstyle{remark}
\newtheorem*{remark}{Remark}

% Fonts, symbols, languages
\usepackage[utf8]{inputenc}
% \usepackage{ebgaramond}  % ebgaramond, palatino, fourier
\usepackage{enumerate}  % Custom numbered items, such as steps 1, 2, etc.
\usepackage{enumitem}  % Ordinals
\usepackage{moreenum}
\usepackage{marginnote}
\usepackage{verbatim}
\usepackage{marginnote}
\usepackage{multicol}

% References and links
\usepackage{hyperref}
\hypersetup{
    colorlinks=true,
    linkcolor=blue,
    filecolor=magenta,
    urlcolor=cyan
    }
\usepackage{url}

\endinput
