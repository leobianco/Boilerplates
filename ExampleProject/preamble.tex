% Preamble

% Graphics
\usepackage{graphicx}
\usepackage{tikz}
\usepackage{tikz-network}
\usetikzlibrary{
	mindmap, trees, arrows, arrows.meta, positioning, decorations.markings, 
	fit, decorations.pathmorphing
}
\usepackage{caption}
\usepackage{subcaption}  % For grids of images
\usepackage[section]{placeins}  % Figures stay in section
\usepackage[framemethod=tikz]{mdframed} % Colored boxes
\newmdenv[innerlinewidth=0.5pt, roundcorner=4pt,linecolor=red!40,backgroundcolor=red!10,innerleftmargin=6pt,
innerrightmargin=6pt,innertopmargin=6pt,innerbottommargin=6pt]{questionbox}

% Mathematics and code
\usepackage{amsthm,amsmath}
\usepackage{flexisym}  % recommended for breqn
\usepackage{breqn}  % for breaking long equations, load it before amssymb
\usepackage{amssymb, mathrsfs}
\usepackage{mathtools}  % for declaring new paired delimiters
\usepackage{algorithm}
\usepackage{algpseudocode}

\newtheorem{theorem}{Theorem}  % Use options to name theorems
\newtheorem{proposition}{Proposition}
\newtheorem{lemma}{Lemma}  
\theoremstyle{definition}
\newtheorem{definition}{Definition}
\newtheorem{hypothesis}{Hypothesis}
\theoremstyle{remark}
\newtheorem*{remark}{Remark}
\newtheorem*{notation}{Notation}

% Fonts, symbols, languages
\usepackage[utf8]{inputenc}
\usepackage{ebgaramond}  % ebgaramond, palatino, fourier
\usepackage{enumerate}  % Custom numbered items, such as steps 1, 2, etc.
\usepackage{enumitem}  % Ordinals
\usepackage{moreenum}
\usepackage{marginnote}
\usepackage{verbatim}
\usepackage{marginnote}
\usepackage{multicol}
% \setlength{\parindent}{0pt}  % If you do not want paragraph indentation.
%\usepackage[francais]{babel}

% References and links
\usepackage{hyperref}
\hypersetup{
    colorlinks=true,
    linkcolor=blue,
    filecolor=magenta,
    urlcolor=cyan
    }
\usepackage{url}

% Subfile (last)
\usepackage{subfiles}

\endinput % To close this file
